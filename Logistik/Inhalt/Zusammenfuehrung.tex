\section{Zusammenführung der Ergebnisse}
\label{sec:Zusammenfuehrung}

Im Zuge dieser Ausarbeitung wurden sowohl die Möglichkeiten als auch die Grenzen
von selbststeuernden Prozessen in der Produktionslogistik untersucht. Bei den
beiden Untersuchungen wurden jedoch unterschiedliche Zieldimensionen
herangezogen. Die Möglichkeiten wurden hierbei in Bezug auf die Flexibilisierung
der Produktionsprozesse untersucht. Die Grenzen wurden in Bezug auf die
logistische Zielerreichung analysiert, welche sich unter anderem aus der
Termintreue, der mittleren Durchlaufzeit und der Auslastung zusammensetzt.

Bei der Untersuchung der Grenzen wurde herausgestellt, dass der in dieser
Ausarbeitung behandelte Montageprozess aufgrund der geringen Komplexität nur
eine geringe logistische Zielerreichung bei einem hohen Selbststeuerungsgrad
erreicht. Die logistische Zielerreichung bei einer Umsetzung des
Montageprozesses als klassischen Prozess ohne Selbststeuerung ist deutlich
höher. Bei der klassischen Umsetzung kann jedoch nicht von den umfangreichen
Möglichkeiten profitiert werden, welche für den selbststeuernden Prozess
herausgestellt wurden. Es empfiehlt sich daher den Montageprozess mit einem
geringen Maß an Selbststeuerung zu realisieren. Hierdurch sinkt die logistische
Zielerreichung im Vergleich zu dem klassischen Prozess ohne Selbststeuerung nur
marginal. Es kann jedoch zumindest teilweise von den herausgestellten
Möglichkeiten der Selbststeuerung profitiert werden.

Hieraus wird deutlich, dass sich die Entscheidung, mit welchem Maß an
Selbststeuerung ein produktionslogistischer Prozess umgesetzt werden sollte,
nicht allgemeingültig treffen lässt. Es handelt sich bei dieser Entscheidung um
eine Einzelfallentscheidung, die von den Anforderungen an den konkreten
Produktionsprozess abhängig gemacht werden muss.
