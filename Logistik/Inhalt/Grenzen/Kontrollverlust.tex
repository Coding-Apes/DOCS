\subsection{Kontrollverlust bei vollständiger Selbststeuerung}
\label{sec:GrenzenKontrollverlust}

Als letzter Aspekt der Grenzbetrachtung wird eine vollständig autonome
Produktionssteuerung betrachtet. Aus der vorgestellten Abbildung wird
ersichtlich, dass die Autoren Scholz-Reiter, de Beer, Böse und Windt bei
vollständiger Selbststeuerung eine geringe logistische Zielerreichung
voraussagen. Dieser Meinung stimmen wir ebenfalls mit nachfolgender Begründung
zu:

Bei einer vollständigen Autonomie der Produktionsprozesse in einem
heterarchischen System bleibt keine Kontrollmöglichkeit bzw. zentrale
Eingriffsmöglichkeit mehr offen. Das Fehlen einer zentralen
Eingriffsmöglichkeit führt bei Änderungen von Entscheidungsparametern zu
produktionslogistischen Fehlern. \hfill \\
Es wird dazu zunächst angenommen, dass die Auswahl der PKW-Rücklichtblende im
vorgestelltem Fallbeispiel von einer Information abhängt, die an den
Rücklichtern angebracht ist. Ändert sich dieser Entscheidungsparameter, das
heißt die Entscheidung wird beispielsweise aufgrund einer anderen Information
getroffen, muss diese Veränderung jeder autonomen Entscheidungseinheit separat
mitgeteilt werden. Aufgrund der fehlenden Möglichkeit diese Änderung zentral zu
verbreiten, entscheidet jede Einheit zunächst falsch. Diese falschen
Entscheidungen können dazu führen, dass viele Produkte fehlerbehaftet
produziert werden oder die Produktion angehalten werden muss. Eine vollständige
Autonomie dezentraler Einheiten ist aus diesen Gründen nicht sinnvoll.

\clearpage