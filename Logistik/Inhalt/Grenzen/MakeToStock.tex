\subsection{Fehlender Mehrwert bei Make-to-Stock-Produktionen}
\label{sec:GrenzenMakeToStock}

Produktionsabläufe bei denen die Produkte auf Vorrat produziert werden, werden als "`make-to-stock"' bezeichnet. Bei dieser 
Produktionsform ist der Mehrwert von selbststeuernden Prozessen aus unserer Sicht nicht relevant. 
Diese Meinung begründen wir mit dem fehlenden Bedarf nach Flexibilität in dieser Produktionsform.
Analog zur Annahme im vorherigen Kapitel kann an dieser Stelle angenommen werden, dass das PKW-Rücklicht als ein 
Massenprodukt in einem Push-Prozess\footnotemark auf Vorrat produziert werden. 
Die Produktionsmengen von diesen make-to-stock-Produkten 
werden anhand von deterministischen Marktkennzahlen geplant und hängen damit nicht von variablen Kundenwünschen ab. Der 
Produktionsablauf kann vorrausgeplant werden und muss nicht flexibel auf Änderungen reagieren. Ein gesteigerter Bedarf 
nach Flexibilität ist nicht vorhanden.
Darüber hinaus führt planerische Sicherheit der Produktion dazu, dass der Mehrwert der optimalen Maschinenauslastung bei 
selbststeuernden Prozessen auch mit zentraler Steuerung erreicht werden kann. Die optimale Auslastung aller Maschinen kann 
in diesem Beispiel auf Grund der vorher geplanten Produktionsmenge zentral berechnet und gesteuert werden.

\footnotetext{Erläuterung Push-Prozess}

Am Ende dieser Betrachtung bleibt aber festzuhalten, dass selbsteuernde Prozesse bei komplexen make-to-stock-Produktionen 
eine Erhöhung der Ausfallsicherheit mit sich bringen können. Dieser vergleichsweise geringe Mehrwehrt rechtfertigt aber 
unserer Sicht nicht den Aufwand, der für eine Umstellung auf eine selbststeuernde Produktion anfällt.
