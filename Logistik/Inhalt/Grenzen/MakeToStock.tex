\subsection{Fehlender Mehrwert bei Make-to-Stock-Produktionen}
\label{sec:GrenzenMakeToStock}

Produktionsabläufe bei denen die Produkte auf Vorrat produziert werden, werden
als "`Make-to-Stock"' bezeichnet. Bei dieser Produktionsform ist der Mehrwert von
selbststeuernden Prozessen aus unserer Sicht nicht relevant. Diese Meinung
begründen wir mit dem fehlenden Bedarf an Flexibilität in dieser
Produktionsform.\hfill \\An dieser Stelle wird angenommen, dass das
PKW-Rücklicht als ein Massenprodukt in einem Push-Prozess\footnote{Produktion auf Vorrat ohne
vorherige Initiierung durch Kundenbedarf} auf Vorrat produziert wird. Die
Produktionsmengen von diesen Make-to-Stock-Produkten werden anhand von
deterministischen Marktkennzahlen geplant und hängen damit nicht von variablen
Kundenwünschen ab. Der Produktionsablauf kann vorausgeplant werden und muss
nicht flexibel auf Änderungen reagieren. Ein gesteigerter Bedarf nach
Flexibilität ist nicht vorhanden.

Darüber hinaus führt planerische Sicherheit der Produktion dazu, dass der
Mehrwert der optimalen Maschinenauslastung bei selbststeuernden Prozessen auch
mit zentraler Steuerung erreicht werden kann. Die optimale Auslastung aller
Maschinen kann in diesem Beispiel auf Grund der vorher geplanten
Produktionsmenge zentral berechnet und gesteuert werden.
\clearpage
Am Ende dieser Betrachtung bleibt aber festzuhalten, dass selbsteuernde
Prozesse bei komplexen Make-to-Stock-Produktionen eine Erhöhung der
Ausfallsicherheit mit sich bringen können. Dieser vergleichsweise geringe
Mehrwehrt rechtfertigt aber unserer Sicht nicht den Aufwand, der für eine
Umstellung auf eine selbststeuernde Produktion anfällt.
