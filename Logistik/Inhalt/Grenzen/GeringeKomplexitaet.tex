\subsection{Fehlender Mehrwert bei geringer Komplexität}
\label{sec:GrenzenKomplexitaet}

Zunächst kann aus der dargestellten Abbildung entnommen werden, dass die Autoren
Scholz-Reiter, de Beer, Böse und Windt der Meinung sind, dass der Mehrwehrt von
Selbststeuerung für logisitische Prozesse mit abnehmender Komplexität der
Produktionsprozesse ebenfalls abnimmt. Bei einfachen logistischen Prozessen
übersteigt der Aufwand der Implementierung und des Betriebes von autonomen
Produktionsanlagen den Nutzen, den diese mit sich bringen. Bezogen auf das
eingangs erwähnte Fallbeispiel lässt sich diese Meinung einfach verdeutlichen:

Angenommen die Produktion der PKW-Rücklichter ist eine reine Massenproduktion
aus wenigen Einzelteilen und ohne besondere Variantenvielfalt, dann kann die
Produktion als wenig komplex eingestuft werden. Die Steuerung der
Produktionsanlagen weist in diesem Fall eine ebenso geringe Komplexität auf, da
die Eingangsgrößen des Prozesses deterministisch geplant werden können. Wir sind
der Meinung, dass in einem solchen System die Mehrwehrte von selbststeuernden
Prozessen, im Gegensatz zu zentral gesteuerten Prozessen, nicht signifikant
sind.

Diese Meinung begründen wir mit dem fehlenden Bedarf nach autonomer
Entscheidungsfindung. Da die Steuerung der Produktionsanlagen wenig komplex ist
kann sie auch von zentralen Systemen übernommen werden. Ein
Geschwindigkeitsvorteil bei der autonomen Entscheidungsfindung bleibt aus. Auch
die Steuerung der Produktionswege zur Optimierung der Maschinenauslastung und
die Reaktion auf Maschinenausfälle kann durch eine zentrale Steuerung
zeitgerecht realisiert werden.
