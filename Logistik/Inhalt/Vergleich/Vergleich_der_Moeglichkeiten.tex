\subsection{Vergleich der Möglichkeiten von selbststeuernden Prozessen mit der klassischen Fließfertigung}
\label{sec:Vergleich_der_Moeglichkeiten}

Der selbststeuernde Prozess und der klassische Fließfertigungsprozess besitzen
unterschiedliche Fertigungsweisen sowie unterschiedliche Strukturen. Dennoch
stellen beide Prozesse das gleiche Produkt her. Es soll herausgestellt werden,
worin sich die beiden Prozesse in der Fertigung von PKW-Rücklichtern
unterscheiden. Die im \verweis{Moeglichkeiten} erarbeiteten Möglichkeiten der
selbststeuernden Prozesse werden hierbei auf den klassischen Prozess bezogen.
%TODO: Soll hier noch Punkt 3 durch einen Verweis ersetzt werden?

\paragraph{Vermeidung von Stillstandszeiten}\hfill \\
Innerhalb der Betrachtung der Möglichkeiten wurde das Verhalten beim Ausfall
einer Bearbeitungsstation in selbststeuernden Prozessen beschrieben. Die
Produktion wird weitergeführt und die Produkte werden zu einem späteren
Zeitpunkt an der ausgefallenen Bearbeitungsstation bearbeitet. Wird dieser
Ausfall einer Bearbeitungsstation innerhalb des klassischen Prozesses der
Fließfertigung betrachtet, ergibt sich ein Stillstand der gesamten Produktion.
Dieser begründet sich durch die strenge lineare Reihenfolge der Montageschritte
innerhalb des Fertigungsprozesses. Fällt hierbei eine Maschine innerhalb der
Fertigungsstraße aus, so wirkt sich dieser Ausfall unmittelbar auf die gesamte
Fertigung aus. Die gesamte Produktion kann in diesem Falle erst wieder
aufgenommen werden, wenn der Ausfall behoben wurde. Wir erkennen hierraus, dass
selbststeuernde Prozesse besser als statische Prozesse  auf den Ausfall einer
Bearbeitungsstation reagieren können.

\paragraph{Fertigung von mehreren Varianten zur Laufzeit} \hfill \\
Eine weitere erkannte Möglichkeit der selbststeuernden Prozesse ist die
Fertigung von verschiedenen Varianten innerhalb einer Fertigungsstraße. Hierbei
werden die im Produkt verbauten RFID-Chips als Informationsträger für die
unterschiedlichen Varianten genutzt. Innerhalb der Fließfertigung werden den
einzelnen Produkten hingegen keine Informationen mitgegeben und es werden
keine individuellen Produkte unterschieden. Um die Fertigung von verschiedenen
Varianten umsetzen zu können, wird ein Produktionsplan erstellt.
Bedingt durch diesen Plan werden an den Bearbeitungsstationen die richtigen
Module pro Produkt verbaut. Somit kann nur eine Variante gleichzeitig
produziert werden. Anhand dieses Nachteils im Vergleich zum selbststeuernden
Prozess erkennen wir, dass bei einer Produktion von mehreren Varianten der
selbststeuernde Prozess flexibler reagieren kann.



\paragraph{Schnelle Reaktion auf sich ändernde Kundenwünsche/
Marktsituationen} \hfill \\
Die erkannte Möglichkeit der selbststeuernden Prozesse schnell auf Kundenwünsche
reagieren zu können, wurde durch das Zusammenspiel von Softwareagenten und
RFID-Chips je Produkt umgesetzt. Diese sind in der Lage die Variation eines
herzustellenden Produktes noch innerhalb der Produktion, je nach Auftrag,
abzuändern. Innerhalb der klassischen Fließfertigung ist eine Änderung der
Variante während der Produktion nicht mehr möglich. Die zu produzierende
Variante sowie die Anzahl wird, wie beschrieben, über den vorher erstellten
Produktionsplan festgelegt. So ist es nicht möglich diese Variante zur
Laufzeit der Produktion noch anzupassen, um auf Kundenwünsche oder
Marktveränderungen zu reagieren. Um große Mengen an Ausschussware verhindern zu
können, wäre nur ein Stoppen der gesamten Produktion im Fließprozess möglich.
Durch diese starre Abarbeitung des Produktionsplans ist ein klarer Nachteil des
klassischen Prozesses im Bezug auf kurzfristige Kundenwünsche erkennbar.

\paragraph{Lastverteilung innerhalb der Produktion} \hfill \\
Die Lastverteilung innerhalb des selbststeuernden Prozesses wurde ebenfalls als
Möglichkeit herausgestellt. Im selbststeuernden Prozess können die Produkte die
Reihenfolge der Arbeitsschritte je nach Maschinenauslastung selbst wählen. Eine
Neuanordnung der Reihenfolge zur Bearbeitung würde im klassischen
Fließfertigungsprozess den Umbau der gesamten Produktionstraße bedeuten. Dies
wäre aufgrund eines hohen Zeitaufwandes nicht denkbar und unpraktikabel. Wird
nun zusätzlich angenommen, dass die Länge der Montageschritte der einzelnen
Bearbeitungsstationen unterschiedlich ist, wird sich, bedingt durch die
Fließfertigung, eine Leerlaufzeit vor einem kürzeren Montageschritt ergeben.

Verdeutlicht wird dies anhand des nachfolgenden Beispiels. Angenommen die
Montage der Dichtung dauert, aufgrund des etwas höheren Aufwandes, 30 Sekunden
und der darauf folgende Fertigungsschritt, die Montage der Leuchtmittel, 20
Sekunden. So würde beim klassischen Prozess ein Rohbauteil durch die
Dichtungsmontage und danach durch die Leuchtmittelmontage transportiert werden,
worauf das nächste Rohbauteil direkt folgen und in die Dichtungsmontage fahren
würde. Nachdem die Fertigungsstation mit der Leuchtmittelmontage nach 20
Sekunden fertig ist, benötigt die Dichtungsmontage noch zehn Sekunden zur
Fertigstellung. Danach erst würde das Rohbauteil zur nächsten Station fahren.
Somit wären zehn Sekunden bei jedem Rohbauteil nicht effektiv genutzt worden,
da die Leuchtmittelmontage diese Zeit auf das nächste Teil wartet.
Innerhalb des selbststeuernden Prozesses können diese Wartezeiten durch die
dynamische Reihenfolge der Bearbeitungsschritte vermieden werden.

Es wurden die erkannten Möglichkeiten der selbststeuernden Prozesse mit einem
klassischen Prozess, welcher das gleiche Produkt herstellt, verglichen. Keine
der erarbeiteten Möglichkeiten konnte innerhalb des statischen Prozesses
erkannt werden. Anhand dieses Ergebnisses erkennen wir, dass die Fähigkeiten,
die der selbststeuernde Prozess im Punkt Flexibilität bei kurzfristigen
Kundenwünschen sowie die Dynamik beim kurzfristigen Maschinenausfall aufweist,
nicht vom klassischen Prozess abgebildet werden kann. Ein weiterer
interessanter Aspekt, der sich innerhalb der Untersuchung herausgestellt hat,
ist die Minimierung der Durchlaufzeit. Hierbei spielen mehrere spezielle
Fähigkeiten des selbststeuernden Prozesses ineinander. Zum einen wird durch die
dynamische Reihenfolge der Bearbeitung die Last auf die einzelnen
Arbeitsschritte verteilt. Zum anderen fallen Rüstzeiten durch die Möglichkeit
der zeitgleichen Produktion von mehreren Variationen weg.

Durch die genannten Möglichkeiten wird deutlich, dass eine Selbststeuerung der
Prozesse die Produktion verbessert und einige Vorteile bietet, die die
klassische Fließfertigung nicht aufweist. Trotz der offensichtlichen Vorteile
der selbststeuernden Prozesse, setzen nur 15\% der deutschen Unternehmen diese
ein.\footnote{\url
http://www.ipl-mag.de/scm-praxis/330-industrie-40-ausblick-in-die-praxis} Ein
möglicher Grund für diese Tatsache könnte aus unserer Sicht sein, dass die
meisten Unternehmen noch davor zurückschrecken. Eine Umstrukturierung der
betrieblichen Fertigungsprozesse zur Selbststeuerung hin, bedeutet für die
Unternehmen einen hohen Aufwand sowie lange Planungs- und Einführungszeiten.
 
Eine Umstrukturierung bei komplexen Fertigungsprozessen stellt dementsprechend
einen zeit- und aufwandsintensiven Vorgang dar. Dieser hohe Aufwand resultiert
in ebenso hohen Kosten. Hieraus schließen wir, dass die meisten Unternehmen
entweder den hohen Aufwand sowie die Kosten scheuen oder selbststeuernde
Prozesse aufgrund von simpler Produktionsprozesse als nicht notwendig
ansehen.
\clearpage