\section{Fazit}
\label{sec:Fazit}

Innerhalb der Ausarbeitung wurden die Möglichkeiten selbststeuernder Prozesse
anhand eines Beispiels herausgearbeitet und mit einer vergleichbaren
Fließproduktion verglichen. Es wurde untersucht, wie sich die beiden
Produktionsprozesse, besonders im Punkt Flexibilität und Dynamik, verhalten. Es
wurde deutlich,dass eine Produktion mit selbststeuernden Prozessen deutlich
besser auf kurzfristige Änderungen reagieren kann. Der Grund hierfür ergibt
sich besonders durch die flexible Wahl der Reihenfolge der
Bearbeitungsschritte. Dennoch konnte im Punkt 5 festgestellt werden, dass es
auch Grenzen von selbststeuernden Prozessen gibt und diese nicht als
Allzwecklösung angesehen werden können. Ist die Komplexität der Produkte sehr
gering und werden Produkte auf Vorrat produziert, können die Möglichkeiten aus
den selbststeuernden Prozessen nicht voll ausgenutzt werden.

Ein kontrovers zu diskutierender Aspekt ist der drohende Kontrollverlust
innerhalb einer komplett selbststeuernden Produktion. Wir vertreten hier den
Standpunkt, dass Maschinen den semantischen Zusammenhang von Produkten nicht
erkennen können. Es werden so Produkte mit offensichtlichen Fehlern als
fehlerfrei produziert erkannt. Denkbar wären hier zum Beispiel ein Fehler in
der Beschriftung von Produkten. Eine Maschine würde diesen Fehler nicht
bemerken, da die Beschriftung nicht von ihr interpretiert werden kann. Somit
erkennen wir die Thesen hinter der Abbildung im Punkt 5 von den Autoren
Scholz-Reiter, de Beer, Böse und Windt als korrekt an.

Innerhalb dieser Ausarbeitung wurde nur sehr kurz auf die zu entstehenden
Mehrkosten bei der Implementierung eines selbststeuernden Prozesses im Punkt
4.2 eingegangen. Es wird aber deutlich, dass die Umsetzung einer Produktion mit
Selbststeuerung deutlich kostenintensiver ist, als die Produktion mit einem
klassischen Prozess. Deswegen ist es wichtig, dass immer die Komplexität und
gewünschte Dynamik innerhalb einer Produktion beachtet wird. So ist es nicht
sinnvoll bei einer reinen Make-to-Stock-Produktion ohne kurzfristige Änderungen
und mit geringer Produktionskomplexität auf selbststeuernde Prozesse zu setzen.

Bei, wie in der Einleitung beschriebenen Produkten, die vom Kunden individuell
angepasst werden können, ist eine Produktion mit selbststeuernden Prozessen
durch die gewünschte Flexibilität und Dynamik denkbar.
%TODO: MKV mark den Satz nicht ich finde er macht die Arbeit am Ende rund oder
% das sollte er machen :D

%TODO: Hier werden auch Kapitel einfach mit Punkt X beschrieben ist das so ok ?
%TODO: Fußzeilen im Gesamtem Dokument nachsehen -> Quellen einbauen!