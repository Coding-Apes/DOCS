\section{Fazit}
\label{sec:Fazit}
Innerhalb der Ausarbeitung wurden die Möglichkeiten selbststeuernder Prozesse
anhand eines Beispiels herausgearbeitet und mit einer vergleichbaren
Fließproduktion verglichen. Hierbei wurde untersucht, wie sich die beiden
Prozesse in bestimmten Situationen verhalten und welche Besonderheiten zu
erkennen sind. So wurden die Möglichkeiten und Grenzen selbststeuernder
Prozesse herausgestellt. Hierbei sind wir während der Untersuchung auf zwei
unterschiedliche Ergebnisse gekommen. Einerseits bieten selbststeuernde
Prozesse wichtige Vorteile in der Produktion,wie Lastverteilung, die Fertigung
mehrerer Varianten zur Laufzeit oder die schnelle Reaktionsfähigkeit bei
Änderung von Kundenwünschen. Andererseits muss abgewägt werden, ob es in jedem
Produktionsprozess sinnvoll ist Selbststeuerung, im Bezug auf die logistische
Zielerreichung, einzusetzen.Hierbei sollten Faktoren wie Kosten und Aufwand bei
der Einführung der Selbststeuerung betrachtet werden und in die Entscheidung
für oder gegen selbststeuernde Prozesse mit einfließen. So bieten
selbststeuernde Prozesse im Bezug auf die logistische Zielerreichung nicht
immer die optimale Lösung, sondern ist dies von der Komplexität des jeweiligen
Produktionsprozesses abhängig. Daher stellt die Entscheidung für oder gegen
Selbststeuerung für jedes Unternehmen eine Einzelfallentscheidung dar.

Zusammenfassend können wir sagen, dass die Nutzung selbststeuernder Prozesse
jedes Unternehmen individuell für sich entscheiden und hierbei alle relevanten
Faktoren zur Entscheidungsfindung einbeziehen muss, um die größtmögliche
Zielerreichung mit maximalen Vorteilen zu erreichen.


%TODO: MKV mark den Satz nicht ich finde er macht die Arbeit am Ende rund oder
% das sollte er machen :D

%TODO: Hier werden auch Kapitel einfach mit Punkt X beschrieben ist das so ok ?
%TODO: Fußzeilen im Gesamtem Dokument nachsehen -> Quellen einbauen!