\section{Fazit}
\label{sec:Fazit}

Innerhalb der Ausarbeitung wurden die Möglichkeiten selbststeuernder Prozesse
anhand eines Beispiels herausgearbeitet und mit einer vergleichbaren
Fließproduktion verglichen. Es wurde untersucht, wie sich die beiden
Produktionsprozesse, besonders im Punkt Flexibilität und Dynamik, verhalten. Es
wurde deutlich, dass eine Produktion mit selbststeuernden Prozessen deutlich
besser auf kurzfristige Änderungen reagieren kann. Der Grund hierfür ergibt
sich besonders durch die flexible Wahl der Reihenfolge der
Bearbeitungsschritte. Dennoch konnte im \verweis{Grenzen} festgestellt werden,
dass es auch Grenzen von selbststeuernden Prozessen gibt und diese nicht als
Allzwecklösung angesehen werden können. Ist die Komplexität der Produkte sehr
gering oder werden Produkte auf Vorrat produziert, können die Möglichkeiten aus
den selbststeuernden Prozessen nicht voll ausgenutzt werden.

Ein kontrovers zu diskutierender Aspekt ist der drohende Kontrollverlust
innerhalb einer komplett selbststeuernden Produktion. Wir vertreten hier den
Standpunkt, dass eine vollständige Autonomie dezentraler Einheiten nicht
sinnvoll ist, da zentrale Eingriffs- und Kontrollmöglichkeiten fehlen.
Somit erkennen wir die Thesen hinter der Abbildung im \verweis{Grenzen} von den
Autoren Scholz-Reiter, de Beer, Böse und Windt als korrekt an.

Innerhalb dieser Ausarbeitung wurde nur sehr kurz auf die zu entstehenden
Mehrkosten bei der Implementierung eines selbststeuernden Prozesses im
\verweis{Vergleich_der_Moeglichkeiten} eingegangen. Es wird aber deutlich, dass
die Umsetzung einer Produktion mit Selbststeuerung deutlich kostenintensiver
ist, als die Produktion mit einem klassischen Prozess. Deswegen ist es
wichtig, dass immer die Komplexität und gewünschte Dynamik innerhalb einer Produktion
beachtet wird. So ist es nicht sinnvoll bei einer reinen
Make-to-Stock-Produktion ohne kurzfristige Änderungen und mit geringer
Produktionskomplexität auf selbst\-steu\-ern\-de Prozesse zu setzen.

Wir haben die Betrachtung von selbststeuernden Prozessen mit dem Bedarf nach
individuellen Produktwünschen eingeleitet. Am Ende dieses Essays können wir
festhalten, dass eine individuelle Massenproduktion mit dem Ansatz von
selbststeuernden Prozessen effektiv möglich ist.

%TODO: MKV mark den Satz nicht ich finde er macht die Arbeit am Ende rund oder
% das sollte er machen :D

%TODO: Hier werden auch Kapitel einfach mit Punkt X beschrieben ist das so ok ?
%TODO: Fußzeilen im Gesamtem Dokument nachsehen -> Quellen einbauen!