\section{Grenzen selbststeuernder produktionslogistischer Prozesse}
\label{sec:Grenzen}

Im vorherigen Kapitel wurde die Möglichkeiten von selbststeuernden logistischen Prozessen herausgestellt. Diesen 
Möglichkeiten stehen aber auch Grenzen der Selbststeuerung gegenüber, die wir im Folgenden herausstellen.

Die Grenzbetrachtung basiert auf \abbildung{Grenzbetrachtung} aus \citet{evolution2007}:

\begin{figure}[htb] 
\centering
\includegraphics[width=1.0\textwidth]{Grenzbetrachtung.png}
\caption[Grenzbetrachtung]{Grenzen selbststeuernder Prozesse\protect\footnotemark}
\label{fig:Grenzbetrachtung}
\end{figure}
\footnotetext{entnommen aus \citet[S.~3]{evolution2007}}

Die Autoren stimmen mit der hier dargestellten Grenzbetrachtung überein. Die dargestellten Grenzen lassen sich dabei in 
zwei Mängel unterscheiden, die bei bestimmten Produktionssituationen vorliegen. Zum einen lässt sich aus unserer Sicht ein 
fehlender Mehrwert bei geringer  Produktionskomplexität oder Make-to-Stock-Produktion und zum anderen ein möglicher 
Kontrollverlust als Grenzen der Selbststeuerung anführen. Diese Meinung wird in den nachfolgenden Abschnitten erläutert.

\subsection{Fehlender Mehrwert bei geringer Komplexität}
\label{sec:GrenzenKomplexitaet}

Zunächst kann aus der dargestellten Abbildung entnommen werden, dass die Autoren
Scholz-Reiter, de Beer, Böse und Windt der Meinung sind, dass der Mehrwehrt von
Selbststeuerung für logisitische Prozesse mit abnehmender Komplexität der
Produktionsprozesse ebenfalls abnimmt. Bei einfachen logistischen Prozessen
übersteigt der Aufwand der Implementierung und des Betriebes von autonomen
Produktionsanlagen den Nutzen, den diese mit sich bringen. Bezogen auf das
eingangs erwähnte Fallbeispiel lässt sich diese Meinung einfach verdeutlichen:

Angenommen die Produktion der PKW-Rücklichter ist eine reine Massenproduktion
aus wenigen Einzelteilen und ohne besondere Variantenvielfalt, dann kann die
Produktion als wenig komplex eingestuft werden. Die Steuerung der
Produktionsanlagen weist in diesem Fall eine ebenso geringe Komplexität auf, da
die Eingangsgrößen des Prozesses deterministisch geplant werden können. Wir sind
der Meinung, dass in einem solchen System die Mehrwehrte von selbststeuernden
Prozessen, im Gegensatz zu zentral gesteuerten Prozessen, nicht signifikant
sind.

Diese Meinung begründen wir mit dem fehlenden Bedarf nach autonomer
Entscheidungsfindung. Da die Steuerung der Produktionsanlagen wenig komplex ist
kann sie auch von zentralen Systemen übernommen werden. Ein
Geschwindigkeitsvorteil bei der autonomen Entscheidungsfindung bleibt aus. Auch
die Steuerung der Produktionswege zur Optimierung der Maschinenauslastung und
die Reaktion auf Maschinenausfälle kann durch eine zentrale Steuerung
zeitgerecht realisiert werden.

\subsection{Fehlender Mehrwert bei Make-to-Stock-Produktionen}
\label{sec:GrenzenMakeToStock}

Produktionsabläufe bei denen die Produkte auf Vorrat produziert werden, werden
als "`Make-to-Stock"' bezeichnet. Bei dieser Produktionsform ist der Mehrwert von
selbststeuernden Prozessen aus unserer Sicht nicht relevant. Diese Meinung
begründen wir mit dem fehlenden Bedarf an Flexibilität in dieser
Produktionsform. An dieser Stelle wird angenommen, dass das PKW-Rücklicht als
ein Massenprodukt in einem Push-Prozess\footnote{Produktion auf Vorrat ohne
vorherige Initiierung durch Kundenbedarf} auf Vorrat produziert wird. Die Produktionsmengen von diesen Make-to-Stock-Produkten werden anhand von
deterministischen Marktkennzahlen geplant und hängen damit nicht von variablen
Kundenwünschen ab. Der Produktionsablauf kann vorausgeplant werden und muss
nicht flexibel auf Änderungen reagieren. Ein gesteigerter Bedarf nach
Flexibilität ist nicht vorhanden.

Darüber hinaus führt planerische Sicherheit der Produktion dazu, dass der
Mehrwert der optimalen Maschinenauslastung bei selbststeuernden Prozessen auch
mit zentraler Steuerung erreicht werden kann. Die optimale Auslastung aller
Maschinen kann in diesem Beispiel auf Grund der vorher geplanten
Produktionsmenge zentral berechnet und gesteuert werden.

Am Ende dieser Betrachtung bleibt aber festzuhalten, dass selbsteuernde
Prozesse bei komplexen Make-to-Stock-Produktionen eine Erhöhung der
Ausfallsicherheit mit sich bringen können. Dieser vergleichsweise geringe
Mehrwehrt rechtfertigt aber unserer Sicht nicht den Aufwand, der für eine
Umstellung auf eine selbststeuernde Produktion anfällt.

\subsection{Kontrollverlust bei vollständiger Selbststeuerung}
\label{sec:GrenzenKontrollverlust}

Als letzter Aspekt der Grenzbetrachtung wird eine vollständig autonome
Produktionssteuerung betrachtet. Aus der vorgestellten Abbildung wird
ersichtlich, dass die Autoren Scholz-Reiter, de Beer, Böse und Windt bei
vollständiger Selbststeuerung eine geringe logistische Zielerreichung
voraussagen. Dieser Meinung stimmen wir ebenfalls mit nachfolgender Begründung
zu:

Bei einer vollständigen Autonomie der Produktionsprozesse in einem
heterarchischen System bleibt keine Kontrollmöglichkeit bzw. zentrale
Eingriffsmöglichkeit mehr offen. Das Fehlen einer zentralen
Eingriffsmöglichkeit führt bei Änderungen von Entscheidungsparametern zu
produktionslogistischen Fehlern. \hfill \\
Es wird dazu zunächst angenommen, dass die Auswahl der PKW-Rücklichtblende im
vorgestelltem Fallbeispiel von einer Information abhängt, die an den
Rücklichtern angebracht ist. Ändert sich dieser Entscheidungsparameter, das
heißt die Entscheidung wird beispielsweise aufgrund einer anderen Information
getroffen, muss diese Veränderung jeder autonomen Entscheidungseinheit separat
mitgeteilt werden. Aufgrund der fehlenden Möglichkeit diese Änderung zentral zu
verbreiten, entscheidet jede Einheit zunächst falsch. Diese falschen
Entscheidungen können dazu führen, dass viele Produkte fehlerbehaftet
produziert werden oder die Produktion angehalten werden muss. Eine vollständige
Autonomie dezentraler Einheiten ist aus diesen Gründen nicht sinnvoll.

\clearpage