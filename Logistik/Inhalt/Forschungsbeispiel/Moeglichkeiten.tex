\subsection{Möglichkeiten von selbststeuernden Prozessen}
\label{sec:Moeglichkeiten}

Die innerhalb des Beispiels erkannten Möglichkeiten im Bezug auf die Fertigung
von PKW-Rücklichtern, werden nun konkret betrachtet. Hierbei werden
unterschiedliche Sichtweisen der Betrachtung angewendet. So gibt es die
Auftragssicht und die Ressourcensicht.\footnote{\citet[S.~234]{arnold2008}}
Die Auftragssicht beschreibt dabei die Struktur eines Auftrages und dessen Durchlauf in der Wertschöpfungskette. Die
Ressourcensicht bildet hingegen den Materialfluss innerhalb der Produktion ab.
Unter Beachtung dieser beiden Sichten ergeben sich folgende Möglichkeiten aus
dem vorher dargestellten Beispiel:

\paragraph{Schnelle Reaktion auf sich ändernde Kundenwünsche /
Marktsituationen} \hfill \\
Im Beispiel ist es möglich sehr schnell auf sich ändernde Kundenwünsche oder
Marktsituationen innerhalb der Produktion zu reagieren. Sollte der Absatz einer
Produktvariante oder die Bestellung eines Kunden sich zur Laufzeit der Montage
ändern, kann auf diese neue Situation sofort reagiert werden. Durch die bereits
erwähnten RFID-Chips in den Produkten können die Softwareagenten die Produkte
innerhalb der Produktion reorganisieren. Dies bedeutet, dass alle Produkte
angepasst an den neuen Auftrag gefertigt werden. Konkret werden halbfertige
Produkte vom stornierten Auftrag übernommen und mit den gewünschten neuen
Parametern des aktuellen Auftrags fertiggestellt. Die Abhängigkeiten des
vorgestellten Variantenkorridors sind hierbei zu beachten. So könnten Produkte,
die bereits mit weißen Leuchtmitteln ausgestattet sind, nicht mehr mit einer
durchsichtigen oder schwarzen Blende versehen werden. Wenn im aktuellen Auftrag
keine PKW-Rücklichter mit weißen Leuchtmitteln gefordert sind, entsteht in
diesem Fall eine kleine Menge an Ausschussware. Diese könnte in einem
nachfolgenden Auftrag aber weiter verarbeitet werden.

\paragraph{Lastverteilung} \hfill \\
Innerhalb des Beispiels ist eine Lastverteilung unter den einzelnen
Bearbeitungstationen möglich. Ist beispielsweise die Bearbeitungsstation der
Kabelbauminstallation besetzt, entsteht keine Wartezeit für ein anderes Produkt.
Dieses wird unter Beachtung des Variantenkorridors an einer anderen freie
Arbeitsstation weitergeleitet, d.h. es wird beispielsweise zuerst das
Leuchtmittel oder die Dichtung montiert, bevor der Kabelbaum verlegt wird.
Denkbar wäre es auch, dass somit die unterschiedlichen Bearbeitungszeiten der
Stationen ausgeglichen werden.
%TODO: Punkt wird zu kurz Angesprochen??

\paragraph{Vermeidung von Stillstandszeiten}\hfill \\
Sollte in dem dargestellten Beispiel eine Station kurzfristig ausfallen, so
leiten sich Produkte selbstständig auf eine alternative Station um. Hierbei 
werden, genau wie bei der Lastverteilung, die Abhängigkeiten aus dem
Variantenkorridor beachtet. Die gesamte Produktion würde, mit Ausnahme der
ausgefallenen Arbeitsstation, weitergeführt werden. Während andere
Montageschritte weiterhin durchlaufen werden, wird die Montagestation instand
gesetzt. Sobald der Fehler behoben ist, kann die Station wieder in Betrieb
genommen werden. Diese gliedert sich daraufhin automatisch in den Arbeitsablauf
ein und kann von den Bauteilen erneut durchlaufen werden.

\paragraph{Fertigung von mehreren Varianten zur Laufzeit} \hfill \\
Mit dem vorgestellten Beispielprozess ist es möglich verschiedene Varianten des
PKW-Rücklichtes zeitgleich in einer Produktionsstraße zu produzieren. Die
Information, welche Module im Produkt verbaut werden, ist direkt an jedem
einzelnen Rücklicht gespeichert. Hierzu ist in jedem Rücklicht ein RFID-Chip
eingelassen. Die Informationen werden an den einzelnen Bearbeitungsstationen
eingelesen und der Maschine / dem Arbeiter angezeigt. Es wird so garantiert,
dass jedes Produkt der Bestellung entspricht. Hiermit ergibt sich die
Möglichkeit in dem Beispielprozess alle drei Varianten gleichzeitig
herzustellen. Denkbar wäre, dass sich das Rücklicht der Variante 1 innerhalb
der Lampenmontage befindet und das Rücklicht der Variante 2 zeitgleich in der
Blendenmontage.

