\section{Einleitung}
\label{sec:Einleitung}
Heutzutage wünschen sich Kunden immer mehr Produkte ihren eigenen Vorstellungen
anzupassen und individuell gestalten zu können. Um diesen Kundenwünschen gerecht
zu werden, ist es notwendig sich als Unternehmen diesem dynamischen Änderungen
anzupassen.  Der Bekleidunghersteller Nike sowie der Autohersteller Opel haben
diese Entwicklung erkannt und bieten ihren Kunden bei bestimmten Produktmodellen
die Möglichkeit zur Individualisierung.\footnote{Nike Konfigurator NIKEiD:
\url{http://www.nike.com/de/de_de/c/nikeid/}} So bietet Opel zum Beispiel bei
seinem Automodell „Adam“ die individuelle Konfiguration nach speziellen
Kundenwünschen an.\footnote{Opel Adam Konfigurator:
\url{http://konfigurator.opel-adam.de/}} Dort ist es möglich das Auto in den
Bereichen Motor, Außen- und Innendesign, Sonderausstattung, etc. zu
konfigurieren und in dieser individuellen Konfiguration zu bestellen. Aus der
so entstehenden Variantenvielfalt der Produkte resultiert eine gesteigerte
Komplexität im Produktionsprozess.

Ein Ansatz, die Flexibilität trotz der gestiegener Komplexität zu gewährleisten,
ist die Einführung selbststeuernder Prozesse. Dieser Ansatz wurde in dem
Sonderforschungsbereich 637 der Universität Bremen in einem Forschungsprojekt
thematisiert. In dem Projekt wurde die Selbststeuerung von Prozessen am Beispiel
des Produktionsprozesses von PKW-Rücklichtern näher untersucht. An diesem Punkt
setzt diese Arbeit an und beschreibt die Möglichkeiten der selbststeuernden
Prozesse anhand dieses Fallbeispiels.

Ziel ist es den Prozess aus dem Fallbeispiel mit der Realisierung in einem
klassischen Fließfertigungsprozess zu vergleichen, um Möglichkeiten und Grenzen
von selbststeuernden Prozessen im Vergleich zur Fließfertigung zu
identifizieren.

Um zunächst ein einheitliches Verständnis in diesem Bereich zu schaffen, wird
die Definition des selbststeuernden Prozesses erläutert. Aufbauend auf der
theoretischen Definition wird das bereits erwähnte Forschungsbeispiel der
Universität Bremen aufgegriffen und näher betrachtet. Anhand des Beispiels
werden zunächst die Möglichkeiten der selbststeuernden Prozesse in der
Produktionslogistik aufgezeigt. Im Nachhinein wird ein klassischer Prozess
modelliert, mit dem der selbststeuernde Prozess verglichen wird, um Vorteile
beider Prozesse aufzuzeigen. Hierauf aufbauend werden die Grenzen von
selbststeuernden Prozessen erläutert. Zum Schluss werden in einem abschließenden
Fazit die gewonnenen Erkenntnisse in dieser Ausarbeitung reflektiert.

\clearpage
%ToDo: Link auf Nike noch anpassen