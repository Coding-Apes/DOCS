\section{Einleitung}
\label{sec:Einleitung}

Heutzutage wünschen sich Kunden immer mehr Produkte ihren eigenen Wünschen anzupassen 
und individuell gestalten zu können. Um diesen Kundenwünschen gerecht zu werden, ist es 
notwendig sich als Unternehmen diesem dynamischen Umfeld anzupassen. Unternehmen wie Nike
oder Opel haben dies erkannt und bieten ihren Kunden bei bestimmten Produktmodellen die 
Möglichkeit zur Individualisierung\footnote{Opel Adam Konfigurator: \url{http://konfigurator.opel-adam.de/}}.
So bietet Opel zum Beispiel bei seinem Automodell „Adam“ die individuelle Konfiguration 
nach speziellen Kundenwünschen an. Dort ist es möglich das Auto nach Motor, Außen- und Innendesign, 
Sonderausstattung, etc. zu konfigurieren und zu bestellen. Durch die so entstehende Variantenvielfalt 
resultiert eine gesteigerte Komplexität der Produktionsprozesse, die trotzdem flexibel gestaltet 
werden müssen. Um diese Flexibilität trotz der gesteigerten Komplexität zu gewährleisten, gibt es 
den Ansatz der selbststeuernden Prozesse, die dieses realisieren sollen. So hat sich bereits die 
Universität Bremen anhand eines Forschungsprojektes mit diesem Thema befasst. In dem Projekt wurde 
die Selbststeuerung von Prozessen anhand des Produktionsprozesses von Rücklichtern für Autos näher 
untersucht. Genau dort setzt diese Arbeit an und soll die Möglichkeiten und Grenzen der selbststeuernden 
Prozesse anhand des Forschungsbeispiels herausarbeiten.

Im Nachfolgenden wird die Vorgehensweise in dieser Ausarbeitung genannt. Um zunächst ein einheitliches 
Verständnis in diesem Bereich zu gewährleisten, wird definiert, was genau ein selbststeuernder Prozess ist. 
Aufbauend auf den theoretischen Inhalten wird das bereits genannte Forschungsbeispiel aufgegriffen und 
näher betrachtet. Anhand des Beispiels werden zunächst die Möglichkeiten der selbststeuernden Prozesse 
in der Produktionslogistik aufgezeigt. Im Nachhinein werden analog zu den Möglichkeiten die Grenzen 
dieser Prozesse aufgezeigt. Zum Schluss sollen in einem abschließenden Fazit die gewonnenen Erkenntnisse 
in dieser Ausarbeitung reflektiert werden.
