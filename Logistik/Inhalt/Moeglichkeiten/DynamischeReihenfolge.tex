\subsection{Dynamische Reihenfolge der Fertigungsprozesse}
\label{DynamischeReihenfolge}

Innerhalb des vorgestellten Fertigungsprozesses kann zwischen den Montageschritte des Rücklichtes im gewissen Maß variiert 
werden. Diese Varierungsmöglichkeiten werden innerhalb des in der Beispiel Erläuterung beschriebenen 
“Produkt-Varianten-Korridor”s beschrieben. Unter Berücksichtigung der erläuterten Abhängigkeiten 
können Produkte dynamisch im Fertigungsprozess wandern. Somit kann auf Störungen innerhalb des 
Fertigungsprozesses reagiert werden\footnote{Vgl. Buch ?}. Wenn innerhalb des  Beispiel eine Mschine kurzfristig 
ausfallen sollte wird hierauf reagiert und die Produkte werden innerhalb eines anderen 
Arbeitsschritten weiter bearbeitet. Sobald die defekte Maschine wieder Einsatzbereit ist, 
wird die Bearbeitung der Produkte fortgesetzt. Die gesamte Produktion wird hierbei 
nicht durch den Ausfall der einen Maschine still gelegt. 
Im Vergleich zur Just-In-Sequence Bearbeitung bietet die Dynamische Reihenfolge auch vorteile. 
Innerhalb der Just-In-Sequence methode müssen alle zu verbauenden Module in der richtigen 
Reihenfolge an Produktionsband geliefert werden, um die richtige Kombination innerhalb des 
fertigen Produktes zu Gewehrleisten. Nehmen wir nun an es wird, bedingt durch einen Produktfehler 
die Reihenfolge der Produkte auf dem Produktionsband geändert. So müssen alle weiter zu verbauenden 
Module an die neue Anordnung angepasst werden. 
Durch die dynamische Reihenfolge der Fertigungsprozesse und der Tatsache das die Produkte selbst 
Informationen über deren Bearbeitung enthalten ist diese Problem nicht bei selbsteuerden Prozessen gegeben
