\section{Definition selbststeuernder Prozesse}
\label{sec:Definition}

\begin{quote}
“Selbststeuerung beschreibt Prozesse dezentraler Entscheidungsfindung in
heterarchischen Strukturen. Sie setzt voraus, dass interagierende Elemente in
nichtdeterministischen Systemen die Fähigkeit und Möglichkeit zum autonomen
Treffen von Entscheidungen besitzen.”\footnote{\citet[S.~42]{huelsmann2005}}
\end{quote}

Diese logistische Definition nennt als zentrales Merkmal von selbststeuernden
Prozessen, dass die einzelnen Elemente im Produktionsprozess selbstständig
Entscheidungen fällen können. Dieser eigenständige Entscheidungsprozess kann
dadurch ermöglicht werden, dass der gesamte Prozess eine heterarchische Struktur
besitzt und jedes Element damit selbst entscheidungsberechtigt ist. Dies bewirkt
eine Dezentralisierung des Entscheidungsprozesses. Die Entscheidung wird also
nicht mehr vom Gesamtsystem, sondern von den einzelnen Systemelementen
selbstständig getroffen. Die Entscheidungen jedes einzelnen Systemelements
basieren dabei auf Informationen und Eigenschaften, die jedes Produkt besitzt.
Anhand dieser Informationen wird, bei Eintreten eines nicht geplanten
Systemzustandes, entschieden, welche Aktion als nächstes durchgeführt werden
soll. Der Eintritt von nicht geplanten Zuständen des Gesamtsystems, wird als
Nicht-Determinismus verstanden. „Unter Nicht-Determinismus wird verstanden, dass
trotz genauester Messung des Systemzustandes und der Kenntnis aller wirkenden
Systemgesetze keine Vorhersage des Systemverhaltens über einen längeren
Zeitraum möglich ist.“\footnote{\citet[S.~10]{boese2012}}
 