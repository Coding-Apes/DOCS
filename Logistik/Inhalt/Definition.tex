\section{Definition}
\label{sec:Definition}

\begin{quote}
“Selbststeuerung beschreibt Prozesse dezentraler Entscheidungsfindung in heterarchischen Strukturen. 
Sie setzt voraus, dass interagierende Elemente in nichtdeterministischen Systemen die Fähigkeit und 
Möglichkeit zum autonomen Treffen von Entscheidungen besitzen.”\footnote{XX}
\end{quote}

Diese Definition der selbststeuernden Prozesse sagt aus, dass ein zentrales Merkmal der Selbststeuerung 
ist, dass die einzelnen Elemente im Produktionsprozess selbstständig Entscheidungen fällen können. 
Dieser eigenständige Entscheidungsprozess kann dadurch ermöglicht werden, dass der gesamte Prozess 
eine nicht hierarchische Struktur besitzt und jedes Element damit selbst entscheidungsberechtigt ist. 
Dadurch wird eine Dezentralisierung des Entscheidungsprozesses vom Gesamtsystem auf die einzelnen 
Systemelemente generiert. Die Entscheidungen jedes einzelnen Systemelements basieren dabei auf Informationen 
und Eigenschaften, die jedes Element besitzt. Anhand dieser wird, bei Eintreten eines nicht geplanten 
Systemzustandes des Gesamtsystems, eine Entscheidung  gefällt. Der Eintritt von nicht geplanten Zuständen 
des Gesamtsystems ist dabei der nicht-deterministischen Eigenschaft geschuldet.
"`Unter Nicht-Determinismus wird verstanden, dass trotz genauester Messung des Systemzustandes 
und der Kenntnis aller wirkenden Systemgesetze keine Vorhersage des Systemverhaltens über einen längeren Zeitraum 
möglich ist."'\footnote{\citet[S.~15]{boese2012}}